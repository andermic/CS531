\documentclass{article}
\usepackage[pdftex]{graphicx}
\usepackage{amsmath}
\author{Michael Anderson}
\title{Homework Set 5}
\begin{document}
\maketitle
\newpage
\begin{enumerate}
\item[\textbf{7.1}]

The set of possible models for the presence of pits and a wumpus in [1,3],[2,2], and [3,1] is:\\

\begin{tabular}{ c c c c c c c l }

& P in [1,3] & P in [2,2] & P in [3,1] & W in [1,3] & W in [2,2] & W in [3,1] & Notes\\
1 & N & N & N & N & N & N & $\alpha_2$ \\
2 & N & N & N & N & N & Y & $\alpha_2$ \\
3 & N & N & N & N & Y & N & $\alpha_2$ \\
4 & N & N & N & Y & N & N & $\alpha_2$, $\alpha_3$ \\
5 & N & N & Y & N & N & N & $\alpha_2$ \\
6 & N & N & Y & N & N & Y & $\alpha_2$ \\
7 & N & N & Y & N & Y & N & $\alpha_2$ \\
8 & N & N & Y & Y & N & N & $KB$, $\alpha_2$, $\alpha_3$ \\
9 & N & Y & N & N & N & N & \\
10 & N & Y & N & N & N & Y & \\
11 & N & Y & N & N & Y & N & \\
12 & N & Y & N & Y & N & N & $\alpha_3$ \\
13 & N & Y & Y & N & N & N & \\
14 & N & Y & Y & N & N & Y & \\
15 & N & Y & Y & N & Y & N & \\
16 & N & Y & Y & Y & N & N & $\alpha_3$ \\
17 & Y & N & N & N & N & N & $\alpha_2$ \\
18 & Y & N & N & N & N & Y & $\alpha_2$ \\
19 & Y & N & N & N & Y & N & $\alpha_2$ \\
20 & Y & N & N & Y & N & N & $\alpha_2$, $\alpha_3$ \\
21 & Y & N & Y & N & N & N & $\alpha_2$ \\
22 & Y & N & Y & N & N & Y & $\alpha_2$ \\
23 & Y & N & Y & N & Y & N & $\alpha_2$ \\
24 & Y & N & Y & Y & N & N & $\alpha_2$, $\alpha_3$ \\
25 & Y & Y & N & N & N & N & \\
26 & Y & Y & N & N & N & Y & \\
27 & Y & Y & N & N & Y & N & \\
28 & Y & Y & N & Y & N & N & $\alpha_3$ \\
29 & Y & Y & Y & N & N & N & \\
30 & Y & Y & Y & N & N & Y & \\
31 & Y & Y & Y & N & Y & N & \\
32 & Y & Y & Y & Y & N & N & $\alpha_3$ \\

\end{tabular}

Since in all models where $KB$ is true (only model 8), $\alpha_2$
and $\alpha_3$ are also true, $KB \models \alpha_2$ and $KB \models \alpha_3$.

\item[\textbf{7.2}]

It is given that:

$1: mythical \Rightarrow immortal$\\
$2: \neg mythical \Rightarrow (mortal \wedge mammal)$\\
$3: (immortal \vee mammal) \Rightarrow horned $\\
$4: horned \Rightarrow magical$\\

To attempt to prove $mythical$, assume $\neg mythical$ and look for a
contradiction. By 2, this leads to $mortal$ and $mammal$, which leads to
$horned$ and $magical$ by 3 and 4. Since no further inferences can be made, and
since no contradiction is derived, it is not provable that the unicorn is
mythical.

Since $mythical$ is either true or false, by 1 and 2 $immortal \vee (mortal
\wedge mammal)$. By 3, this implies $horned$, and then by 4 this implies
$magical$. It is provable that the unicorn is magical and horned.

\item[\textbf{7.6}]

\begin{enumerate}
\item[a)]
True. By the property of monotonicity for KB's, adding to a KB cannot
reduce the set of statements that it entails. It is given that $\alpha \models
\gamma$ or $\beta \models \gamma$. If $\alpha \models \gamma$, then adding
$\beta$ to $\alpha$ does not reduce its power to entail $\gamma$. Similarly, if
$\beta \models \gamma$, then adding $\alpha$ does not reduce $\beta$'s power.

\item[b)]
True. By the truth table of $and$, knowing $(\beta \wedge \gamma)$ implies separately
that $\beta$ is true and $\gamma$ is true. So any KB that entails $(\beta \wedge
\gamma)$ also entails them separately.

\item[c)]
False. Given any arbitrary knowledge base, it can be said that the price of tea in
China is either some arbitrary value $p$, or it is not. But if the knowledge base has nothing to do
with tea whatsoever, than it does not entail that the price of tea in China is
$p$, and it does not entail that it isn't either.

\end{enumerate}

\item[\textbf{7.10}]
\begin{enumerate}
\item[a)]
Valid because implications cannot be false when their left and right sides have
the same truth value.

\item[b)]
Neither. Not valid because if $Smoke$ is true and $Fire$ is false, the
sentence is false. Not unsatisfiable because the sentence is satisfied if for
example $Smoke$ and $Fire$ are both true.

\item[c)]
Neither. Not valid because the sentence is false if $Smoke$ is false and $Fire$
is true. Not unsatisfiable because the sentence is satisfied if for example
both $Smoke$ and $Fire$ are true.

\item[d)]
Valid because $Fire \vee \neg Fire$ is always true, and anything or'ed with
that is true also.

\item[e)]
Valid because the sentence is true for all combinations of values of the three
variables.

\item[f)]
Valid because the sentence is true for all combinations of values of the three
variables.

\item[g)]
Valid because the sentence is true if any of $Big$, $Dumb$, or $Big \Rightarrow
Dumb$ are true, and $Big \Rightarrow Dumb$ can only be false when $Big$ is true.
\end{enumerate}

\item[\textbf{7.22}]
\begin{enumerate}
\item[a)]
$(X_{1,2} \wedge X_{2,1}) \vee (X_{1,2} \wedge X_{2,2}) \vee (X_{2,1} \wedge X_{2,2})$

\item[b)]
Make a disjunctive clause for each of the n choose k combinations of squares,
and $or$ them together. Each combination represents one of the possible
placements of k mines into the n neighbors. To turn this disjunctive form into
CNF, use De Morgan's Laws and/or the distributive law.

\item[c)]
Use the process described in b) for every currently visible, numbered square on
the board. Put all CNF clauses generated into DPLL. If DPLL returns a
definitive value for a given square, than that square for certain does or does
not contain a mine depending on the value returned. If it does not, then it is
not provable that the square contains or does not contain a mine

\item[d)]
Since there are N choose M ways to place M mines into N squares, the number of
clauses at any give point is $O(N choose M)$. 

\item[e)]
If DPLL in c) does not return a definitive value for a square, it may still be
provable that a square does not contain a mine due to the global constraint.

\item[f)]
\end{enumerate}

\item[\textbf{7.26}]
$FacingEast^{t+1} = (FacingEast^t \wedge (\neg TurnRight^t \wedge \neg TurnLeft^t))$ \\
$ \vee (FacingSouth^t \wedge TurnLeft^t)$ \\
$ \vee (FacingNorth^t \wedge TurnRight^t)$ \\

$WumpusAlive^{t+1} = WumpusAlive^t \wedge \neg(Shoot^t \wedge Scream^{t+1}$)

\end{enumerate}
\end{document}
